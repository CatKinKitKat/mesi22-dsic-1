\chapter{Introdução}
\section{Apresentação do tópico e do objetivo do trabalho}
\section{Contextualização e alcance do papel}
\subsection{Importância do tema}

\chapter{Revisão da literatura}
\section{Principais descobertas e avanços na área}
\section{Desafios atuais}
\subsection{Leis internacionais}
\subsubsection{Tratados internacionais}
\subsubsection{Recomendações e orientações da União Europeia e da ONU}
\subsection{Leis nacionais}
\subsubsection{Lei de Cibersegurança e de Cibercrime de Portugal}
\subsubsection{Outras leis nacionais relevantes}
\subsection{Tendências tecnológicas}
\subsubsection{Inovações e tecnologias emergentes}
\subsubsection{Impacto nas questões de cibersegurança e cibercrime}

\chapter{Desafios e oportunidades}
\section{Principais questões e desafios atuais}
\section{Oportunidades na área da cibersegurança e do cibercrime}
\subsection{Proteção de dados pessoais}
\subsubsection{Desafios e oportunidades}
\subsubsection{Propostas de solução}
\subsection{Segurança de sistemas de informação}
\subsubsection{Desafios e oportunidades}
\subsubsection{Propostas de solução}
\subsection{Combate à cibercriminalidade}
\subsubsection{Desafios e oportunidades}
\subsubsection{Propostas de solução}
\subsection{Ameaças cibernéticas}
\subsubsection{Desafios e oportunidades}
\subsubsection{Propostas de solução}

\chapter{Conclusão}
\section{Síntese dos principais pontos do trabalho}
\section{Direções futuras de pesquisa e desenvolvimento}
\section{Reflexão sobre os avanços e desafios atuais e seu impacto no futuro da área}
