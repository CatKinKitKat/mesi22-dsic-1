\chapter{Introdução}

A cibersegurança e o cibercrime são problemas cada vez mais importantes no mundo digital de hoje, e Portugal não é exceção. Uma vez que o país continua a depender da tecnologia e da internet em todas as áreas da sociedade, é fundamental compreender a situação atual da cibersegurança e do cibercrime em Portugal. Pretendemos apresentar neste artigo uma revisão completa das principais tendências, dificuldades e possibilidades do setor, bem como as importantes legislações e tecnologias que pertencem a essas preocupações. Acreditamos que, ao fazê-lo, lançaremos luz sobre o estado atual da cibersegurança e do cibercrime em Portugal e forneceremos um recurso para pessoas interessadas em aprender mais sobre estes assuntos críticos.

\section{Apresentação do tópico e do objetivo do trabalho}

Na sociedade conectada de hoje, a cibersegurança e o cibercrime estão se tornando cada vez mais vitais. A segurança cibernética é a proteção de sistemas e redes de computadores contra acessos ou ataques não autorizados, enquanto o cibercrime é a investigação e repressão de crimes cometidos por meio do uso desses sistemas e redes. Estes campos são especialmente importantes em Portugal devido às consequências, potencialmente graves, de um ciberataque ou violação. Essas implicações podem variar desde a perda de dados críticos até danos à infraestrutura e até a morte. Dada a importância destas preocupações, o objetivo deste estudo é oferecer uma revisão completa do atual nível de cibersegurança e cibercrime em Portugal. Isso envolve investigar as principais descobertas e avanços no assunto, bem como as dificuldades e possibilidades existentes.

Dada a importância destas preocupações, o objetivo deste estudo é oferecer uma revisão completa do atual nível de cibersegurança e cibercrime em Portugal. Isso envolve investigar as principais descobertas e avanços no assunto, bem como as dificuldades e possibilidades existentes, bem como a legislação aplicável e a tecnologia que atendem a essas preocupações. Pretendemos fornecer um recurso completo para a compreensão da situação atual da cibersegurança e do cibercrime em Portugal como resultado disso.

\section{Contextualização e alcance do papel}

O âmbito deste artigo centra-se estreitamente no estado atual da cibersegurança e do cibercrime em Portugal. Para oferecer uma visão completa, veremos as principais descobertas e avanços no assunto, bem como as dificuldades e possibilidades existentes, bem como a legislação e tecnologia aplicáveis. Considere o contexto nacional no qual essas preocupações surgem, bem como quaisquer leis e tratados internacionais aplicáveis. Acreditamos que, ao fazê-lo, conseguiremos fornecer uma visão geral completa da situação atual da cibersegurança e do cibercrime em Portugal, bem como fornecer um excelente recurso para os indivíduos interessados em aprender mais sobre estas questões vitais.


\subsection{Importância do tema}

No mundo digital de hoje, a importância da cibersegurança e do cibercrime em Portugal não pode ser subestimada. À medida que a confiança da sociedade na tecnologia e na internet cresce, é fundamental que as medidas necessárias sejam implementadas para se defender contra-ataques cibernéticos. Trata-se de proteger sistemas e redes de computadores contra acessos não autorizados ou ataques, bem como investigar e processar crimes cometidos por meio do seu uso.

Um ataque ou violação cibernética pode ter sérias implicações, desde a perda de dados confidenciais até danos à infraestrutura e até a morte. Como resultado, é fundamental que Portugal implemente as salvaguardas adequadas para se defender e as suas redes de tais ameaças, bem como o quadro legal para responsabilizar aqueles que cometem crimes cibernéticos.
