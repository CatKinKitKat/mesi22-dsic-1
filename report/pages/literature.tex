\chapter{Revisão da literatura}

Nesta parte, examinaremos as descobertas e avanços mais importantes nas disciplinas de cibersegurança e cibercrime em Portugal, bem como as questões atuais que confrontam estes assuntos. Isto envolve a investigação das mais recentes pesquisas e avanços no setor, bem como as preocupações e tendências importantes que definem a situação atual da cibersegurança e do cibercrime em Portugal.

Queremos dar uma visão geral completa do estado atual do assunto e um recurso útil para as pessoas interessadas em aprender mais sobre esses tópicos vitais, completando um estudo minucioso da literatura. Além disso, veremos a legislação e tecnologia relevantes de cibersegurança e cibercrime em Portugal, bem como são utilizadas para resolver estas questões. Isso oferecerá uma visão geral completa do estado atual da disciplina, bem como da sua trajetória futura.

\section{Principais descobertas e avanços na área}

Ao longo dos anos, houve várias descobertas e avanços no domínio da cibersegurança e do cibercrime em Portugal. Entre os desenvolvimentos significativos estão:

\begin{itemize}
  \item A implementação do Regulamento Geral de Proteção de Dados (GDPR), que melhorou a consciencialização sobre a proteção de dados pessoais e os direitos dos indivíduos em relação aos seus dados pessoais. O RGPD tem uma influência considerável nas empresas portuguesas, obrigando-as a estabelecer maiores medidas de segurança de dados e a serem mais transparentes nas suas operações de tratamento de dados.
  \item A criação da Autoridade Nacional para a Proteção de Dados e Livre Circulação de Dados (CNPD), que tem a seu cargo a aplicação do RGPD em Portugal e o apoio à proteção de dados. A CNPD é fundamental para garantir o cumprimento do RGPD e preservar os direitos dos indivíduos relativamente aos seus dados pessoais.
  \item O estabelecimento de certificações profissionais e programas de treino para auxiliar os indivíduos no desenvolvimento de habilidades e conhecimentos essenciais para atuar na área, bem como a expansão da cibersegurança como carreira. Estes programas permitem aos indivíduos obter conhecimentos e competências específicas em cibersegurança, contribuindo assim para o desenvolvimento de uma força de trabalho robusta e qualificada no setor.
\end{itemize}

Outras grandes conquistas em cibersegurança e cibercrime em Portugal incluem a criação de novas tecnologias e técnicas para combater as ciberameaças e investigar o cibercrime, bem como a adoção das melhores práticas e padrões mundiais do setor.

\section{Desafios atuais}

Apesar da evolução da cibersegurança e do cibercrime em Portugal, ambos os setores continuam a enfrentar vários problemas. Entre os principais desafios estão:

\begin{itemize}
  \item Os perigos cibernéticos estão se tornando mais sofisticados, tornando-os difíceis de identificar e evitar. À medida que as estratégias dos cibercriminosos se tornam mais sofisticadas, a proteção contra esses tipos de ameaças torna-se cada vez mais difícil.
  \item O crescente número de dispositivos e sistemas conectados à Internet aumenta a possibilidade de ataques e violações. A proliferação de dispositivos e sistemas vinculados aumenta drasticamente o perigo de ataques cibernéticos e violações.
  \item A escassez de profissionais competentes para lidar com essas dificuldades, uma vez que a demanda por especialistas em cibersegurança supera significativamente a oferta. Há uma escassez substancial de indivíduos experientes no setor, tornando difícil para as empresas gerir os riscos cibernéticos e investigar o crime cibernético com eficiência.
\end{itemize}

No entanto, ainda há necessidade de legislação e regulamentos mais fortes para se defender contra-ataques cibernéticos, bem como mais educação e conhecimento sobre esses desafios, bem como melhor coordenação e a partilha de informações entre as partes interessadas no campo.

\subsection{Leis internacionais}

Nesta secção, veremos as leis e tratados internacionais de cibersegurança e cibercrime em Portugal, como a Convenção do Conselho da Europa sobre Cibercrime, a Lei de Cibersegurança da União Europeia e o Grupo de Especialistas Governamentais das Nações Unidas sobre Desenvolvimentos no Campo da Informação e Telecomunicações no Contexto da Segurança Internacional (UN GGE). Essas leis e tratados estabelecem uma estrutura para colaboração internacional e fornecem conselhos e diretrizes para resolver os desafios de segurança cibernética e crimes cibernéticos em escala global.

\subsubsection{Tratados internacionais}

Em Portugal, numerosos tratados internacionais abordam questões de cibersegurança e cibercrime. Entre os mais notáveis estão:

\begin{itemize}
  \item A Convenção sobre Cibercrime do Conselho da Europa, o primeiro tratado internacional a abordar o cibercrime e oferece uma estrutura para a cooperação internacional na investigação e punição do cibercrime. Portugal adotou a Convenção, sendo um instrumento fundamental para garantir que os governos tenham as ferramentas e recursos adequados para combater o cibercrime.
  \item A Diretiva de Redes e Sistemas de Informação da UE (Diretiva NIS), que define os requisitos básicos para a segurança de redes e sistemas de informação e obriga os estados-membros a implementar medidas de segurança. A Diretiva NIS é vital para Portugal como membro da UE, pois garante que as redes e os sistemas de informação sejam seguros e possam resistir a ataques cibernéticos.
\end{itemize}

Outros tratados internacionais que dizem respeito à cibersegurança e ao cibercrime em Portugal incluem a Convenção sobre a Utilização de Tecnologias de Informação para Fins Aduaneiros, a Convenção sobre a Proteção de Crianças contra a Exploração Sexual e o Abuso Sexual e a Convenção do Conselho da Europa sobre Branqueamento, Busca e Apreensão, e Confisco dos Produtos do Crime e sobre o Financiamento do Terrorismo. Esses acordos fornecem uma base para a colaboração internacional e orientação sobre como resolver desafios específicos de segurança cibernética e crimes cibernéticos.

\subsubsection{Recomendações e orientações da União Europeia e da ONU}

Além dos tratados internacionais, a União Europeia e as Nações Unidas divulgaram recomendações e diretrizes sobre segurança cibernética e crimes cibernéticos. Aqui estão alguns exemplos:

\begin{itemize}
  \item A Estratégia de Cibersegurança da União Europeia, que descreve uma estratégia abrangente para lidar com a cibersegurança na UE. A Estratégia compreende uma série de etapas destinadas a melhorar a segurança cibernética em toda a UE, como proteger infraestruturas vitais, impulsionar a pesquisa e a inovação e expandir a colaboração internacional.
  \item A União Internacional de Telecomunicações (ITU), que através do seu setor ITU-D faz recomendações sobre segurança cibernética e preocupações com crimes cibernéticos. A ITU-D promove o desenvolvimento de tecnologias de informação e comunicação (TICs) de forma segura, segura e sustentável, e dá conselhos sobre temas como cibercrime, segurança cibernética e segurança infantil online.
\end{itemize}

Estas sugestões e orientações dão orientações vitais para nações como Portugal sobre como gerir os desafios da cibersegurança e do cibercrime, bem como uma base para a colaboração internacional e troca de informações.

\subsection{Leis nacionais}

Nesta parte, veremos as leis nacionais em Portugal que tratam da cibersegurança e do cibercrime, como a Lei Portuguesa de Cibersegurança e Cibercrime e outra legislação nacional relacionada, como a Lei de Proteção de Dados Pessoais e a Lei de Comunicações e Transações Eletrónicas. Estas regras estabelecem um quadro para lidar com as preocupações de cibersegurança e cibercrime em Portugal e garantem que aqueles que cometem crimes cibernéticos enfrentem as consequências.

\subsubsection{Lei de Cibersegurança e de Cibercrime de Portugal}

A Lei de Cibersegurança e Cibercrime (Lei de Cibersegurança e Cibercrime) é uma importante lei nacional de cibersegurança e cibercrime em Portugal. Esta lei de 2016 define o quadro legal para lidar com riscos cibernéticos e crimes em Portugal, bem como proteger as principais infraestruturas e sistemas de informação.

O cibercrime é definido pela lei como qualquer crime cometido utilizando sistemas ou redes de computadores e abrange ofensas como hacking, usurpo de identidade e distribuição de software destrutivo. Também especifica os poderes e responsabilidades dos órgãos de aplicação da lei, como o Ministério Público, a Polícia Judiciária e a Guarda Nacional Republicana. Essas autoridades são responsáveis por investigar e processar crimes cibernéticos, bem como prevenir e mitigar os perigos cibernéticos.

A Lei de Cibersegurança e Cibercrime também estabelece uma série de medidas de cibersegurança, como a construção de um centro nacional de cibersegurança, o desenvolvimento de padrões e diretrizes de cibersegurança e a implementação de um programa de certificação de cibersegurança. A lei também exige troca de informações e colaboração entre várias autoridades e grupos envolvidos em segurança cibernética e crimes cibernéticos.

\subsubsection{Outras leis nacionais relevantes}

Além da Lei de Cibersegurança e Cibercrime, existem inúmeras outras legislações nacionais em Portugal que dizem respeito à cibersegurança e ao cibercrime. Essas leis fornecem uma estrutura para lidar com desafios específicos de segurança cibernética e crimes cibernéticos, e são essenciais para proteger pessoas e empresas contra ameaças e crimes cibernéticos.

Aqui estão alguns exemplos de leis nacionais relevantes:

\begin{itemize}
  \item A Lei de Proteção de Dados Pessoais (Lei de Proteção de Dados Pessoais) rege o tratamento de dados pessoais e define os direitos das pessoas em relação aos seus dados pessoais. Essa lei é fundamental para garantir a segurança dos dados pessoais e dar aos indivíduos controle sobre as suas informações pessoais.
  \item A Lei das Comunicações Eletrónicas (Lei das Comunicações Eletrónicas), que rege a indústria portuguesa de comunicações eletrónicas e inclui leis relativas à segurança das redes e serviços de comunicações eletrónicas. Esta lei é fundamental para garantir a segurança e resiliência das redes e serviços de comunicações eletrónicas face a ataques cibernéticos.
  \item A Lei do Comércio Eletrónico, que rege a comercialização de produtos e serviços pela internet e inclui normas de proteção ao consumidor e de responsabilidade do prestador de serviços. Essa lei é fundamental para proteger os consumidores e responsabilizar as empresas por seu comportamento no comércio pela Internet.
\end{itemize}

\subsection{Tendências tecnológicas}

Esta secção abordará os desenvolvimentos tecnológicos que afetam a cibersegurança e o cibercrime em Portugal, como o crescimento de dispositivos conectados e o uso de inteligência artificial e aprendizado de máquina na cibersegurança. Estes desenvolvimentos trazem possibilidades, mas também dificuldades para a cibersegurança e o cibercrime em Portugal, e continuarão a mudar o panorama destas disciplinas nos próximos anos.

\subsubsection{Inovações e tecnologias emergentes}

O rápido ritmo de inovação técnica e o advento de novas tecnologias é um dos temas importantes no domínio da cibersegurança e do cibercrime em Portugal. Estes avanços e desenvolvimento tecnológico trazem possibilidades, bem como dificuldades, para a cibersegurança e o cibercrime em Portugal.

Alguns exemplos de avanços e desenvolvimento de tecnologia que influenciam essas disciplinas incluem:

\begin{itemize}
  \item AI e aprendizado de máquina são usados para melhorar a precisão e a eficiência dos sistemas de segurança, bem como para identificar e prevenir ataques cibernéticos. Essas tecnologias também são utilizadas para melhorar a investigação e o processo de crimes cibernéticos, mas também representam preocupações significativas, pois os cibercriminosos podem usá-las para automatizar e dimensionar os seus ataques.
  \item A internet das coisas (IoT), que acelera a disseminação de dispositivos e sistemas vinculados, aumentando o risco de ataques e violações. Embora a Internet das Coisas tenha o potencial de aumentar a eficiência e a produção, ela também aumenta o perigo de ataques cibernéticos e violações de dados.
  \item Blockchain é investigada como uma possível opção para aumentar a segurança e a integridade das transações digitais. A tecnologia Blockchain tem o potencial de criar sistemas seguros e descentralizados, imunes a manipulação e fraude, mas também apresenta problemas de aceitação e implementação.
\end{itemize}

\subsubsection{Impacto nas questões de cibersegurança e cibercrime}

Os desenvolvimentos acima mencionados e as novas tecnologias têm uma influência substancial nos desafios da cibersegurança e do cibercrime em Portugal. Essas tecnologias fornecem novas ferramentas e capacidades para identificar e investigar crimes cibernéticos, bem como proteger contra ameaças cibernéticas, mas também oferecem novas dificuldades e fraquezas que devem ser abordadas.

O uso de inteligência artificial e aprendizado de máquina em segurança cibernética, por exemplo, fornece novos recursos para identificar e mitigar ameaças cibernéticas, mas também apresenta novos riscos, pois essas tecnologias podem ser aproveitadas por cibercriminosos para automatizar e dimensionar as suas operações. A proliferação de dispositivos vinculados por meio da IoT aumenta a possibilidade de ataques e violações, mas também abre novas perspetivas para maior eficiência e produtividade. Da mesma forma, o uso da tecnologia blockchain oferece a possibilidade de redes seguras e descentralizadas, mas também apresenta obstáculos em termos de adoção e execução.
